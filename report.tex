\documentclass[11pt,a4paper,twoside]{article}
\usepackage{titlesec}
\usepackage{lipsum}

\begin{document}

\title{\bf Investigating diversity and population structure in Western Chimpanzees using genomic data from fecal samples}
\author{Barney Couch \& Aislyn Taylor}
\date{\today}
\maketitle

\hrule
\bigskip

\begin{abstract}
We analysed a load of DNA samples and hoped to learn something useful from them. Thankfully we learnt a lot, and I think it's safe to say the entire Chimpanzee species is safe given our noble and excellent practical work. Plus we used Latex \cite{lamport94} which means our work is smooth as fuck.
\end{abstract}


\section{Introduction}
This is a small paragraph where we talk about our goals, aims, and more stuff.

\section{Methodology}
\lipsum[2]

\section{Results}

\begin{table}[ht]
\caption{Nonlinear Model Results}
\bigskip
\centering
\begin{tabular}{c c c c}
\hline\hline
Case & Method\#1 & Method\#2 & Method\#3 \\ [0.5ex] % inserts table %heading
\hline
1&50&837&97000 \\
2&47&877&23000\\
3&31&25 & $4.24 \cdot 10^{-4}$ \\
4 & 35 & 144 & 2356 \\
5 & 45 & 300 & 556 \\ [1ex]
\hline
\end{tabular}
\label{table:nonlin}
\end{table}

\lipsum[5]

\section{Conclusion}
\lipsum[4]

\begin{thebibliography}{9}

	\bibitem{lamport94}
	  Leslie Lamport,
	  \emph{\LaTeX: A Document Preparation System}.
	  Addison Wesley, Massachusetts,
	  2nd Edition,
	  1994.

\end{thebibliography}

\end{document}